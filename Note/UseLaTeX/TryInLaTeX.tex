%-*- coding: UTF-8 -*-
% TryInLaTeX.tex
% 自动控制原理

\title{自动控制原理学习笔记}
\author{飞翔的碎碎念}

\documentclass{NHNotebook}

\usepackage{NHPackage}

%\newcolorbox{mybox}[1]{colback=blue!5!white,colframe=blue!75!black,fonttitle=\bfseries,title={#1}}


\begin{document}

\maketitle

\frontmatter

\chapter*{前言}

说两句:

菜狗先看的胡寿松第七版,目前的思路是跟着胡的目录来走的。现在找到了中文版的外国教材《现代控制理论》。
后面会慢慢加国外教材的内容,努力在暑假里多学一点吧。

另外就是作为菜狗本狗,\LaTeX 的使用属于是刚出生级别,这第一份初尝试必然在很多方面具有不足,希望能够获得每一位读者(也许会有吧)对现存问题的指出与建议。

感谢大力支持我的李宇轩同学!    

\tableofcontents

\mainmatter


\part{Modern Control System}

\chapter{控制系统导论}
\section{引言}
摘录
\begin{enumerate}
    \item 系统(对控制工程师而言)$=$ 他们周边环境的一部分
    \item 反馈理论+线性系统理论+综合应用网络理论\&通信理论=\textbf{控制工程}
    \item 控制系统、受控对象(入→受控→出)
    \item 开环控制系统(没有反馈的系统),利用执行机构直接控制受控对象
    \item 闭环反馈控制系统(有反馈的系统)
    \begin{itemize}
        \item 增加了对实际输出的测量,将实际输出与预期输出进行对比
        \item 输出测量值→反馈信号
        \item 比较系统函数,通过比较得到的偏差作为调整控制的依据
        \item 优点:抗干扰+衰减测量噪声    
    \end{itemize}
    \begin{tcolorbox}[reset]
        闭环控制系统对输出进行测量,将此测量信号反馈,并与预期的输入(参考或指令输入)进行比较。
    \end{tcolorbox}
    \item 多回路反馈控制系统
    \begin{itemize}
        \item 更贴合实际情况
        \item 可以是内环$+$外环的结构,内环有专用的一套控制器和传感器来调控监测并调控偏差,外环同理。
        \item 利用单回路系统的思想来学习多回路的系统
    \end{itemize}
    \item 多变量控制系统
    \begin{itemize}
        \item 举个不知道恰不恰当的例子,类似于数学上从一元方程推广到多元方程的感觉。影响系统输出的自变量增加了,系统最终输出的所要变量也增多了。
        \item 比如高炉炼铁,自变量上可以有温度、投料比、炼制时间等,因变量(输出响应)可以有产品纯度、硬度、柔韧性等。
        \item (不知道对不对)
    \end{itemize}
\end{enumerate}

\section{自动控制简史}
不做具体摘录了,看得开心(·v·)✧
\begin{itemize}
    \item 公元前300年\~{}公元前1年之间\phantom{哟}古希腊\phantom{嚯}浮球调节装置
    \item 公认最早应用于工业过程的自动反馈控制器:1769年\phantom{占}瓦特\phantom{位}飞球调节器→控制蒸汽机转速
    \item 1868年\phantom{的}麦克斯韦(J.C.Maxwell)\phantom{字}微分方程→一类调节器模型。重点:研究系统参数对系统性能的影响。
    \item 二战前\phantom{符}伯德(H.W.Bode)、奈奎斯特(H.Nyquist)和布莱克(H.S.Black)\phantom{一}电话系统+电子反馈放大器
    \item ※美国和西欧:频域;前苏联和东欧:时域。
    \item 自动化:对工业过程(加工、制造等)实施自动控制而非人工控制的方法。
    \item 生产率、劳动生产效率
    \item (太多了不抄了自己看当作科普)
    \item 二战期间\phantom{二}自动控制理论及应用→发展高潮。飞机自动驾驶仪、火炮定位系统、雷达天线控制系统等
    \item 20世纪40年代,控制工程发展为工程科学(哇塞)
    \item 二战后\phantom{一}拉普拉斯(Laplace)变换+频域复平面应用→频域方法主导。
    \item 20世纪50年代\phantom{字}s平面方法,esp根轨迹法
    \item 人造卫星和空间时代\phantom{一}时域方法:最优控制理论、鲁棒系统。→控制工程要重视时域和频域。
\end{itemize}

\section{控制系统实例}

摘录
\begin{enumerate}
    \item 控制工程关注分析与设计面向目标的系统。
    \item 现代控制理论关注具有自组织、自学习、自适应、鲁棒性和最优性等特征的系统。
    \item 反馈控制
    \begin{itemize}
        \item 汽车驾驶系统、远洋轮、大型飞机。
        
        分析→测量通过a.视觉和触觉(身体运动)或b.手(传感器)感知方向盘变化来实现

        \item 人工调节容器内的液面高度或位置。
        
        分析→系统的输入(预期输出)是液面参考位置(应该存放在操作手的脑海中);控制放大器为操作手本人;传感器则是操作手视觉。

        \item 类似系统:冰箱(预期/设定的温度)→恒温器(测量实际温度\&设定温度的偏差)→压缩机(起功率放大器作用)\ldots
    \end{itemize}
    \item 反馈控制系统的工业应用
    \begin{itemize}
        \item 机器人(用于替代人劳动的自动化机器,且带有拟人化特征)
        \item 电力工业(需要妥善处理生产过程中各变量间关系以提高产能)
        \item 冶金工业(控制冶炼温度、板材宽/厚/质量等)
        \item 农业
        \item \ldots
        \item (自己看去吧,都是例子,主要是了解原理)
    \end{itemize}
\end{enumerate}

\section{工程设计}
摘录
\begin{enumerate}
    \item 工程设计是工程师的中心工作。\textbf{分析和创新}占据重要地位。
    \begin{tcolorbox}[reset]
        设计就是为达到特定的目的,构思或创建系统的结构、组成和技术细节的过程。
    \end{tcolorbox}
    \item 设计活动规划特定产品/系统的诞生。\begin{enumerate}
        \item 明确用户需求
        \item 论证设计要求
        \item 开发设计方案
        \item 选定解决方案
    \end{enumerate}步骤详见课本p12.

    \item 时间限制是影响设计产品的一个重要因素。
    \item 设计规范/设计要求:对产品或系统将是什么,以及能够做什么的简洁而明确的说明。
    \item 技术指标的4个因素。\begin{itemize}
        \item 设计复杂性
        \item 折中处理
        \item 设计差异
        \item 风险
    \end{itemize}
    \item 分析\&综合
    \begin{quote}
    \textbf{分析:}聚焦眼前事物,通过对物理系统的各种模型的进行细致的考察,得出影响系统稳定性的参数,以确定改进方向。

    \textbf{综合:}侧重于从已有系统中搭建新的、功能更完备的系统。
    \end{quote}
    \item \textbf{设计过程本质是一个不断迭代的过程,但我们始终要找到一个起点。}
    \item 参数分析和优化
    \begin{enumerate}
        \item 参数分析
        \begin{enumerate}
            \item 辨别关键参数
            \item 构建整个系统
            \item 评估系统满足需求的程度
        \end{enumerate}这三步形成了一个迭代循环。
        \item 优化
    \end{enumerate}
\end{enumerate}

\section{控制系统设计}

\begin{tcolorbox}[colback=white,colframe=blue!70!green,colupper=blue!45!black,fonttitle=\bfseries,title=书中章节与控制系统的设计流程模块关系]
    \begin{enumerate}
        \item \textbf{确定控制目标和受控变量,并定义系统性能指标设计要求:}第1章、第3章、第4章和第13章
        \item \textbf{系统定义和建模:}第2章至第4章,第11章至第13章
        \item \textbf{控制系统设计,全系统集成的仿真和分析:}第4章至第13章
    \end{enumerate}
流程图见Markdown。
\end{tcolorbox}
\begin{enumerate}
    \item 第一步:确定系统目标
    \item 第二步:确定需要控制的系统变量
    \item 第三步:拟定技术设计规范/设计要求,一边确定系统变量应该达到的精度指标。
    \begin{itemize}    
        \item 抗干扰能力
        \item 对指令的响应能力
        \item 产生实用执行驱动信号的能力
        \item 灵敏度
        \item 鲁棒性
        \item \ldots
    \end{itemize}以上是应达到的性能。

    \item 第四步:设计出能够实现预期控制性能的系统结构配置。
    系统通常结构配置见课本p2的图1.3,以下是具体包含内容:
    \begin{itemize}
        \item 传感器
        \item 受控对象
        \item 执行机构
        \item 控制器
    \end{itemize}
    \item 第五步:选定执行机构,这与受控对象有关。
    \item 第六步:选择合适的传感器。随后可得到控制系统的这些组成部件的模型。
    →插个眼,建模在这本书前几章会涉及,并回答一些相关的基本问题,让我们了解数学建模的概念。
    \item 第七步:选择控制器。
    \item 第八步:优化系统参数,以便获得所期望的系统性能。
    \item 第九步:重复第一到八步,直到达到预期。
\end{enumerate}

\section{机电一体化系统}
基本要素包括:
\begin{itemize}
    \item 物理系统建模
    \item 传感器与执行机构
    \item 信号与系统←\dag 与控制系统联系紧密
    \item 计算机与逻辑系统
    \item 软件与数据获取   
\end{itemize}
实例有混合动力汽车、风力发电系统、嵌入式计算机(嵌入式控制系统!)
\section{绿色工程}
环保嗯嗯。主要提了绿色系统的基本原则,归纳了绿色工程可应用的五大领域。闲了可以读一读这一节。

\section{控制系统前瞻}
控制系统的目标:使系统具有更高的柔性和自主性。柔性按我的理解就是对环境的适应能力强、可实际应用的领域多,可以实现“一机普适不同环境”。而自主性按照书本里的示例,可以简述为在执行控制程序时不那么需要人类干涉调整,依靠机器自身就可以完成预期工作。

柔性低的改进要点之一是发展计算机视觉(Computer Vision),从软件和硬件上优化机器自身对周围环境的信息捕获与处理能力。另外的改进要点一个是优化人机交互方式,使机器更好地执行人下达的任务,也让人可以更轻松地完成对机器地操控;另一个是改进机器自身的监督控制能力。


\textcolor{black!35!}{\#出于简洁的考虑,省略了1.9设计实例与1.10循序渐进设计实例这两节,以及本章后的对应习题及答案。这两节主要涉及的都是分析设计流程中的某些步骤的具体操作方法,目前笔者急于进入后续内容的学习,故暂不做摘录。(主要还是人菜,开学前多少要学一点老师上课讲的东西,不然肯定挂科)\#}


\chapter{系统数学模型}
%\chapter{自动控制的数学模型}
\section{引言}
理解和控制复杂系统 $\rightarrow$ 获得系统定量数学模型 $\rightarrow$ 仔细分析系统变量间的相互关系,建立数学模型 %$\rightarrow$ 

系统本质上是动态的 $\rightarrow$ 系统行为要用微分方程组进行描述 $\rightarrow$ 如果方程(组)可以线性化 $\rightarrow$ 可用拉普拉斯变换进行简化

系统本身过于复杂 $\rightarrow$ 对系统运动情况做出合理假设

分析研究动态系统的步骤:
\begin{enumerate}
    \item 构建和定义系统及其元件。
    \item 基于基本的物理模型,确定必要的假设条件并推导数学模型。
    \item 列写描述该模型的数学方程(组)。
    \item 求解方程(组),得到所求输出变量的解。
    \item 检查假设条件和所得到的解。
    \item 如果必要,重新分析和设计系统。
\end{enumerate}

\section{物理系统的微分方程(组)}
\begin{enumerate}
    \item 通过型变量
    弹簧例中,作用在弹簧A端的外部扭矩,经由弹簧原封不动地传递到了弹簧B端。此扭矩就是通过型变量。
    \item 跨越型变量
    弹簧例中,弹簧A端和弹簧B端的角速度不相等,经过分别测量,可以得到角速度差。角速度因而称为跨越型变量。
\end{enumerate}

个人理解是,扭矩这个通过型变量,经过弹簧这个系统传递后,未发生传递信息的损失,对这个系统有作用但无实际影响,就像一般路过路人甲或游戏NPC一样,参与但不改变很大的进程;
而角速度这个跨越型变量,经过弹簧传递,传递的信息发生了改变,并且一些重要参数需要经过对其前后差异的对比中得出,就像跨过河流一样,总有些东西是变了。

课本p36.

\textcolor{blue!35}{\#需要补习的一些物理知识:扭矩$T$、磁通匝链数$\lambda_{21}$、角动量$h$、角位移差$\theta_{21}$、流量$Q$}

\textcolor{blue!35}{难绷,还有一些术语也不理解:集总(关联词:线性、动态)。\#}

\begin{enumerate}
    \item 黏性阻尼
    指两个关联变量间呈现线性关系。在质量块-弹簧系统中,摩擦力与质量块的运动速度成正比。(一般是理想情况下会这样)
    \item 干性阻尼(库伦阻尼)
    指两个关联变量间呈现非线性关系。在质量块-弹簧系统中,摩擦力与质量块的运动速度的关系在速度零点附近波动。(一般是结合实际考虑更多因素时会这样)
\end{enumerate}

例:质量块和弹簧

对质量块M作受力分析,有$$F_{\displaystyle{\mbox{合力}}}=F_{M}+F_{f}+F_{\mbox{弹簧}}$$
其中,b是黏性摩擦的摩擦系数,k是弹簧的弹性系数:

\begin{gather*}
    F_{M}=Ma=M \frac{\mathrm{d}^2y(t)}{\mathrm{d}t^2}\\
    F_f=bv=b\frac{\mathrm{d}y(t)}{\mathrm{d}t}\\
    F_{\mbox{弹簧}}=kl=ky(t)
\end{gather*}
从而得到最终的微分方程式:
\[
   M\frac{\mathrm{d}^2y(t)}{\mathrm{d}^2t}+b\frac{\mathrm{d}y(t)}{\mathrm{d}t}+ky(t)=r(t)
\]

%\newcommand*{\dif}{\mathop{}\!\mathrm{d}}
例:RLC电路
利用基尔霍夫电流定律,得到积分-微分方程(PID中的I和D!):
\[
    \frac{v(t)}{R}+C\frac{v(t)}{\dd t}+\frac{1}{L}\int_{0}^{t}v(t)\dd t=r(t)   
\]

\part{自动控制原理}
\chapter{认识自动控制(还有系统)}

\section{自动控制的基本原理与方式}
\subsection{控制系统导论}
系统=周边的环境
\subsection{自动控制技术及其应用}
定义:自动控制≈无人直接操作,依赖机器自身完成人预先设定的任务。

应用:自动化工厂、无人驾驶飞机、无人驾驶汽车、人造卫星着陆其他天体...\dots
\subsection{自动控制科学}
*介绍一下那什么发展史

(偷懒,要看去看Markdown那边的Preview去)

\subsection{反馈控制原理}
\subsubsection{自动控制系统定义}
有控制装置A和被控物体B,以及将二者连接起来的方式C和B的输出量(被控量)D。
A和B借助C的存在,形成一个有机的整体,也就是自动控制系统。

B本身是受A控制的,被控量D是用以衡量B被控制的状态的参考。
D可以是具体的物理量(如温度、压强、速度等),也可以是抽象出来的物理规律或数学规律(如飞行轨迹、记录曲线等)。

A是控制B的所有存在的总体,对B施加的方式C可以有很多很多种。
胡寿松的中文课本里主要介绍最基本的一种,反馈原理。反馈原理需要D的支持。
\subsubsection{分析}
例1:手从桌上取书。

总之用文字语言转述的话,就是:

定义:Input=书的位置,A=眼睛,B=大脑,Output=手与书的相对位置

    (不存在的图图)%\includegraphics{c:\Users\dovegu\AppData\Roaming\Tencent\QQ\Temp\KYRK`6`DMPY`BZ[V{S0I_RN.png}
    
    (流程图用Markdown画了,但是图片一下子弄不过来,先挂起来,以后会用LaTeX画图了再补)(说起来这个图其实更适合用\LaTeX 画呢)

\begin{enumerate}    
    \item 书的位置信息作为输入信号输入到眼睛,经过眼睛处理为生物信号后传递给大脑;
    \item 大脑作为神经中枢,与眼睛、手臂一同承担着控制装置的角色;
    \item 大脑将生物信号经过手臂传递给手,调整手与书籍之间的相对位置;
    \item 手在此时是被控物体,被控量是物理量手与书之间的距离;
    \item 被控量经由眼睛的测量,反馈信息给大脑,由大脑通过对被控对象手的控制,来实现对被控量手与书的距离的调整;
    \item 重复步骤,直到被控量最终达到预定值,完成手取书的动作。
\end{enumerate}

命名:按功能划分,有测量元件(眼)、比较元件(脑)和执行元件(手臂+手)。统称控制装置。

之后介绍的是正负反馈。

负反馈:
(后面龙门刨床的例子自己看吧,不复述了,理解就行,原理和手拿书一样的。
\begin{enumerate}
    \item 元件含义
    \begin{enumerate}
        \item CF:调节触发器,关联电压$u_k$
        \item KZ:给SM供电的晶体管,关联电压$u_a$
        \item TG:测速发电机,关联电压$u_t$,有关系式$u_t\propto n_{实际}$。$u_t$为负反馈电压。
        \item FD:放大器,关联电压$u_0$和$\Delta u$。$u_0$为给定电压,$\Delta u = u_0 - u_t$为偏差电压。由于$\Delta u$较小,无法触发KZ,于是增加FD用于放大
    \end{enumerate}
    \item 负反馈思路
    \begin{itemize}
        \item $u_0 \to n$
        \item $n_{10} \to u_t \to \Delta u \to u_k \to u_a \to n_{11}$
        \item 控制过程(文字描述自己看书):$n_{10}\downarrow \to u_t\downarrow \to \Delta u\uparrow \to u_k\uparrow \to u_a\uparrow \to n_{11}\uparrow$(其中$n_{10}$和$n_{11}$都是SM的实际运行速度)
    \end{itemize}
\end{enumerate}




\emph{切换到 Modern Control System这本书}



\bibliography{math}

\backmatter

\end{document}