%-*- coding: UTF-8 -*-
% TryInLaTeX.tex
% 自动控制原理

\documentclass[oneside,a4paper]{ctexbook}

\title{自动控制原理学习笔记}
\author{飞翔的碎碎念}
\date{\today}

\bibliographystyle{plain}

\usepackage{graphicx}
\usepackage{geometry}

\geometry{left=3cm,right=3cm,top=2cm,bottom=2cm}

\begin{document}

\maketitle

说两句:

菜狗先看的胡寿松第七版,目前的思路是跟着胡的目录来走的。
后面会慢慢加国外教材的内容,努力在暑假里多学一点吧。

另外就是作为菜狗本狗,\LaTeX 的使用属于是刚出生级别,这第一份初尝试必然在很多方面具有不足,希望能够获得每一位读者(也许会有吧)对现存问题的指出与建议。

感谢大力支持我的李宇轩同学!

\tableofcontents

\chapter{认识自动控制}
\section{自动控制的基本原理与方式}
\subsection{自动控制技术及其应用}
定义:自动控制≈无人直接操作,依赖机器自身完成人预先设定的任务。

应用:自动化工厂、无人驾驶飞机、无人驾驶汽车、人造卫星着陆其他天体...\dots
\subsection{自动控制科学}
*介绍一下那什么发展史

(偷懒,要看去看Markdown那边的Preview去)

\subsection{反馈控制原理}
\subsubsection{自动控制系统定义}
有控制装置A和被控物体B,以及将二者连接起来的方式C和B的输出量(被控量)D。
A和B借助C的存在,形成一个有机的整体,也就是自动控制系统。

B本身是受A控制的,被控量D是用以衡量B被控制的状态的参考。
D可以是具体的物理量(如温度、压强、速度等),也可以是抽象出来的物理规律或数学规律(如飞行轨迹、记录曲线等)。

A是控制B的所有存在的总体,对B施加的方式C可以有很多很多种。
胡寿松的中文课本里主要介绍最基本的一种,反馈原理。反馈原理需要D的支持。
\subsubsection{分析}
例1:手从桌上取书。

总之用文字语言转述的话,就是:

定义:Input=书的位置,A=眼睛,B=大脑,Output=手与书的相对位置

    (不存在的图图)%\includegraphics{c:\Users\dovegu\AppData\Roaming\Tencent\QQ\Temp\KYRK`6`DMPY`BZ[V{S0I_RN.png}
    
    (流程图用Markdown画了,但是图片一下子弄不过来,先挂起来,以后会用LaTeX画图了再补)(说起来这个图其实更适合用\LaTeX 画呢)

\begin{enumerate}    
    \item 书的位置信息作为输入信号输入到眼睛,经过眼睛处理为生物信号后传递给大脑;
    \item 大脑作为神经中枢,与眼睛、手臂一同承担着控制装置的角色;
    \item 大脑将生物信号经过手臂传递给手,调整手与书籍之间的相对位置;
    \item 手在此时是被控物体,被控量是物理量手与书之间的距离;
    \item 被控量经由眼睛的测量,反馈信息给大脑,由大脑通过对被控对象手的控制,来实现对被控量手与书的距离的调整;
    \item 重复步骤,直到被控量最终达到预定值,完成手取书的动作。
\end{enumerate}

命名:按功能划分,有测量元件(眼)、比较元件(脑)和执行元件(手臂+手)。统称控制装置。

之后介绍的是正负反馈。

负反馈:
(后面龙门刨床的例子自己看吧,不复述了,理解就行,原理和手拿书一样的。
\begin{enumerate}
    \item 元件含义
    \begin{enumerate}
        \item CF:调节触发器,关联电压$u_k$
        \item KZ:给SM供电的晶体管,关联电压$u_a$
        \item TG:测速发电机,关联电压$u_t$,有关系式$u_t\propto n_{实际}$。$u_t$为负反馈电压。
        \item FD:放大器,关联电压$u_0$和$\Delta u$。$u_0$为给定电压,$\Delta u = u_0 - u_t$为偏差电压。由于$\Delta u$较小,无法触发KZ,于是增加FD用于放大
    \end{enumerate}
    \item 负反馈思路
    \begin{itemize}
        \item $u_0 \to n$
        \item $n_{10} \to u_t \to \Delta u \to u_k \to u_a \to n_{11}$
        \item 控制过程(文字描述自己看书):$n_{10}\downarrow \to u_t\downarrow \to \Delta u\uparrow \to u_k\uparrow \to u_a\uparrow \to n_{11}\uparrow$(其中$n_{10}$和$n_{11}$都是SM的实际运行速度)
    \end{itemize}
\end{enumerate}










\chapter{自动控制的数学模型}


\bibliography{math}


\end{document}